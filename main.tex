%%%%%%%%%%%%%%%%%%%%%%%%%%%%%%%%%%%%%%%
% This is a modified ONE COLUMN version of
% the following template:
% 
% Deedy - One Page Two Column Resume
% LaTeX Template
% Version 1.1 (30/4/2014)
%
% Original author:
% Debarghya Das (http://debarghyadas.com)
%
% Original repository:
% https://github.com/deedydas/Deedy-Resume
%
% IMPORTANT: THIS TEMPLATE NEEDS TO BE COMPILED WITH XeLaTeX
%
% This template uses several fonts not included with Windows/Linux by
% default. If you get compilation errors saying a font is missing, find the line
% on which the font is used and either change it to a font included with your
% operating system or comment the line out to use the default font.
% 
%%%%%%%%%%%%%%%%%%%%%%%%%%%%%%%%%%%%%%
% 
% TODO:
% 1. Integrate biber/bibtex for article citation under publications.
% 2. Figure out a smoother way for the document to flow onto the next page.
% 3. Add styling information for a "Projects/Hacks" section.
% 4. Add location/address information
% 5. Merge OpenFont and MacFonts as a single sty with options.
% 
%%%%%%%%%%%%%%%%%%%%%%%%%%%%%%%%%%%%%%
%
% CHANGELOG:
% v1.1:
% 1. Fixed several compilation bugs with \renewcommand
% 2. Got Open-source fonts (Windows/Linux support)
% 3. Added Last Updated
% 4. Move Title styling into .sty
% 5. Commented .sty file.
%
%%%%%%%%%%%%%%%%%%%%%%%%%%%%%%%%%%%%%%%
%
% Known Issues:
% 1. Overflows onto second page if any column's contents are more than the
% vertical limit
% 2. Hacky space on the first bullet point on the second column.
%
%%%%%%%%%%%%%%%%%%%%%%%%%%%%%%%%%%%%%%

\documentclass[]{deedy-resume-openfont}


\begin{document}

%%%%%%%%%%%%%%%%%%%%%%%%%%%%%%%%%%%%%%
%
%     LAST UPDATED DATE
%
%%%%%%%%%%%%%%%%%%%%%%%%%%%%%%%%%%%%%%
\lastupdated

%%%%%%%%%%%%%%%%%%%%%%%%%%%%%%%%%%%%%%
%
%     TITLE NAME
%
%%%%%%%%%%%%%%%%%%%%%%%%%%%%%%%%%%%%%%


\namesection{Hardik}{Goel}{ \urlstyle{same}\url{}
\href{mailto:goelx033@umn.edu}{goelx033@umn.edu} | 612.532.4261\\
}
\vspace{\topsep}
{\fontspec[Path = fonts/lato/]{Lato-Bol}\selectfont\bfseries\href{https://www.linkedin.com/in/hardikgoel}{https://www.linkedin.com/in/hardikgoel}}\\
\vspace{\topsep}
{\fontspec[Path = fonts/lato/]{Lato-Bol}\selectfont\href{https://github.com/goelhardik/}{https://github.com/goelhardik/}}\\


%%%%%%%%%%%%%%%%%%%%%%%%%%%%%%%%%%%%%%
%
%     EXPERIENCE
%
%%%%%%%%%%%%%%%%%%%%%%%%%%%%%%%%%%%%%%

\section{Experience}

\runsubsection{Cisco Systems}
\descript{| Software Engineer }
\location{July 2011 – August 2015 | Bangalore, India}
\vspace{\topsep} % Hacky fix for awkward extra vertical space
\begin{tightemize}
\item Worked as a firmware engineer on an I/O Virtualized, Converged Network Adapter, which is part of a Data Center solution - UCS (Unified Compute Servers).
\item Worked on iSCSI boot, allowing servers to remotely boot from a central NetApp/EMC
storage array. Resolved timing issues to fix Linux crashes and added support for booting multiple hosts.
\item Worked on board bring-up for new network adapters, programming new chips and adding new network speeds.
\item Created a tool in python for debugging of adapter issues. The tool accepted XML file with commands and generated results using expect scripts.
\item Worked on the re-factoring of CLI based tool for configuration of the adapter from the UCS Manager over the network. This also enabled multi-host configuration on the adapter.
\end{tightemize}
\sectionsep


%%%%%%%%%%%%%%%%%%%%%%%%%%%%%%%%%%%%%%
%
%     EDUCATION
%
%%%%%%%%%%%%%%%%%%%%%%%%%%%%%%%%%%%%%%

\section{Education}
\runsubsection{University of Minnesota}
\descript{| MS in Computer Science}
\location{Expected May 2017 | Twin-Cities, MN } GPA: 4.0 / 4.0\\

\textbullet{} Advanced Algorithms and Data Structures \textbullet{} Matrix Theory
\textbullet{} Artificial Intelligence I
\textbullet{} \\
\textbullet{} Teaching Assistant for undergrad Operating Systems.
\sectionsep

\runsubsection{Indian Institute of Technology, Roorkee}
\descript{| BTech in Electronics \& Communication Engg.}
\location{May 2011 | Roorkee, India}
GPA: 7.6 / 10.0\\
\textbullet{} Digital Signal Processing 
\textbullet{} Digital Communication
\textbullet{} Data Structures and Algorithms
\textbullet{} Mathematics
\textbullet{} Computer Networks
\textbullet{} Computer Architecture
\textbullet{} Database Management
\textbullet{}
\sectionsep

%%%%%%%%%%%%%%%%%%%%%%%%%%%%%%%%%%%%%%
%     PROJECTS
%%%%%%%%%%%%%%%%%%%%%%%%%%%%%%%%%%%%%%

\section{Projects}
\runsubsection{Optical Character Recognition for handwritten Hindi script}
\location{ | \space\space Jan 2011 – May 2011}
Worked in a team of three to develop a complete Optical Character Recognition system for handwritten Hindi script.This involved the use of
techniques of image enhancement, segmentation and feature extraction (adapted to work best for Hindi text). Used Zernike Moments and Center of Mass as features for a static database.\\
\textbullet{} MATLAB \textbullet{}
\sectionsep

\runsubsection{Othello Bot}
\location{ | \space\space December 2015}
Developed an AI to play the game of Othello. Implemented adversary search algorithms such as Minimax and Alpha-Beta Pruning, along with a lot of heuristic based scoring, spcific to Othello.\\
Also made a simple {\fontspec[Path = fonts/lato/]{Lato-Bol}\selectfont\bfseries{\href{http://tictactoeapp.herokuapp.com/}{Tic Tac Toe}}} game for humans to play against an unbeatable bot.\\
\textbullet{} Python \textbullet{} HTML \textbullet{} CSS \textbullet{} Javascript \textbullet{} Flask \textbullet{}
\sectionsep

%%%%%%%%%%%%%%%%%%%%%%%%%%%%%%%%%%%%%%
%     SKILLS
%%%%%%%%%%%%%%%%%%%%%%%%%%%%%%%%%%%%%%

\section{Skills}
\begin{minipage}[t]{.9\textwidth}
\subsection{Programming}
\begin{minipage}[t]{.2\textwidth}
\location{Comfortable:}
\textbullet{} Python \textbullet{} Matlab \textbullet{} 
\end{minipage}
\hfill
\begin{minipage}[t]{.7\textwidth}
\location{Rusty/Familiar:}
\textbullet{} C \textbullet{} Shell \textbullet{} Javascript \textbullet{} CSS \textbullet{} HTML \textbullet{} Lisp \textbullet{} \LaTeX\ \textbullet{}
\end{minipage}
\sectionsep
\end{minipage}

%%%%%%%%%%%%%%%%%%%%%%%%%%%%%%%%%%%%%%
%     AWARDS
%%%%%%%%%%%%%%%%%%%%%%%%%%%%%%%%%%%%%%

\section{Awards}
\vspace{\topsep} % Hacky fix for awkward extra vertical space
\begin{tightemize}
\item Cisco Achievement Program (CAP) award (2x) for identifying and handling release-critical issues.
\item All India Rank 997 in IIT-JEE 2007 (Joint Entrance Examination) among ~300,000 candidates.
\end{tightemize}
\sectionsep
\end{document}  \documentclass[]{article}